% Options for packages loaded elsewhere
\PassOptionsToPackage{unicode}{hyperref}
\PassOptionsToPackage{hyphens}{url}
\PassOptionsToPackage{dvipsnames,svgnames,x11names}{xcolor}
%
\documentclass[
  letterpaper,
]{scrbook}

\usepackage{amsmath,amssymb}
\usepackage{iftex}
\ifPDFTeX
  \usepackage[T1]{fontenc}
  \usepackage[utf8]{inputenc}
  \usepackage{textcomp} % provide euro and other symbols
\else % if luatex or xetex
  \usepackage{unicode-math}
  \defaultfontfeatures{Scale=MatchLowercase}
  \defaultfontfeatures[\rmfamily]{Ligatures=TeX,Scale=1}
\fi
\usepackage{lmodern}
\ifPDFTeX\else  
    % xetex/luatex font selection
\fi
% Use upquote if available, for straight quotes in verbatim environments
\IfFileExists{upquote.sty}{\usepackage{upquote}}{}
\IfFileExists{microtype.sty}{% use microtype if available
  \usepackage[]{microtype}
  \UseMicrotypeSet[protrusion]{basicmath} % disable protrusion for tt fonts
}{}
\makeatletter
\@ifundefined{KOMAClassName}{% if non-KOMA class
  \IfFileExists{parskip.sty}{%
    \usepackage{parskip}
  }{% else
    \setlength{\parindent}{0pt}
    \setlength{\parskip}{6pt plus 2pt minus 1pt}}
}{% if KOMA class
  \KOMAoptions{parskip=half}}
\makeatother
\usepackage{xcolor}
\setlength{\emergencystretch}{3em} % prevent overfull lines
\setcounter{secnumdepth}{5}
% Make \paragraph and \subparagraph free-standing
\ifx\paragraph\undefined\else
  \let\oldparagraph\paragraph
  \renewcommand{\paragraph}[1]{\oldparagraph{#1}\mbox{}}
\fi
\ifx\subparagraph\undefined\else
  \let\oldsubparagraph\subparagraph
  \renewcommand{\subparagraph}[1]{\oldsubparagraph{#1}\mbox{}}
\fi

\usepackage{color}
\usepackage{fancyvrb}
\newcommand{\VerbBar}{|}
\newcommand{\VERB}{\Verb[commandchars=\\\{\}]}
\DefineVerbatimEnvironment{Highlighting}{Verbatim}{commandchars=\\\{\}}
% Add ',fontsize=\small' for more characters per line
\usepackage{framed}
\definecolor{shadecolor}{RGB}{241,243,245}
\newenvironment{Shaded}{\begin{snugshade}}{\end{snugshade}}
\newcommand{\AlertTok}[1]{\textcolor[rgb]{0.68,0.00,0.00}{#1}}
\newcommand{\AnnotationTok}[1]{\textcolor[rgb]{0.37,0.37,0.37}{#1}}
\newcommand{\AttributeTok}[1]{\textcolor[rgb]{0.40,0.45,0.13}{#1}}
\newcommand{\BaseNTok}[1]{\textcolor[rgb]{0.68,0.00,0.00}{#1}}
\newcommand{\BuiltInTok}[1]{\textcolor[rgb]{0.00,0.23,0.31}{#1}}
\newcommand{\CharTok}[1]{\textcolor[rgb]{0.13,0.47,0.30}{#1}}
\newcommand{\CommentTok}[1]{\textcolor[rgb]{0.37,0.37,0.37}{#1}}
\newcommand{\CommentVarTok}[1]{\textcolor[rgb]{0.37,0.37,0.37}{\textit{#1}}}
\newcommand{\ConstantTok}[1]{\textcolor[rgb]{0.56,0.35,0.01}{#1}}
\newcommand{\ControlFlowTok}[1]{\textcolor[rgb]{0.00,0.23,0.31}{#1}}
\newcommand{\DataTypeTok}[1]{\textcolor[rgb]{0.68,0.00,0.00}{#1}}
\newcommand{\DecValTok}[1]{\textcolor[rgb]{0.68,0.00,0.00}{#1}}
\newcommand{\DocumentationTok}[1]{\textcolor[rgb]{0.37,0.37,0.37}{\textit{#1}}}
\newcommand{\ErrorTok}[1]{\textcolor[rgb]{0.68,0.00,0.00}{#1}}
\newcommand{\ExtensionTok}[1]{\textcolor[rgb]{0.00,0.23,0.31}{#1}}
\newcommand{\FloatTok}[1]{\textcolor[rgb]{0.68,0.00,0.00}{#1}}
\newcommand{\FunctionTok}[1]{\textcolor[rgb]{0.28,0.35,0.67}{#1}}
\newcommand{\ImportTok}[1]{\textcolor[rgb]{0.00,0.46,0.62}{#1}}
\newcommand{\InformationTok}[1]{\textcolor[rgb]{0.37,0.37,0.37}{#1}}
\newcommand{\KeywordTok}[1]{\textcolor[rgb]{0.00,0.23,0.31}{#1}}
\newcommand{\NormalTok}[1]{\textcolor[rgb]{0.00,0.23,0.31}{#1}}
\newcommand{\OperatorTok}[1]{\textcolor[rgb]{0.37,0.37,0.37}{#1}}
\newcommand{\OtherTok}[1]{\textcolor[rgb]{0.00,0.23,0.31}{#1}}
\newcommand{\PreprocessorTok}[1]{\textcolor[rgb]{0.68,0.00,0.00}{#1}}
\newcommand{\RegionMarkerTok}[1]{\textcolor[rgb]{0.00,0.23,0.31}{#1}}
\newcommand{\SpecialCharTok}[1]{\textcolor[rgb]{0.37,0.37,0.37}{#1}}
\newcommand{\SpecialStringTok}[1]{\textcolor[rgb]{0.13,0.47,0.30}{#1}}
\newcommand{\StringTok}[1]{\textcolor[rgb]{0.13,0.47,0.30}{#1}}
\newcommand{\VariableTok}[1]{\textcolor[rgb]{0.07,0.07,0.07}{#1}}
\newcommand{\VerbatimStringTok}[1]{\textcolor[rgb]{0.13,0.47,0.30}{#1}}
\newcommand{\WarningTok}[1]{\textcolor[rgb]{0.37,0.37,0.37}{\textit{#1}}}

\providecommand{\tightlist}{%
  \setlength{\itemsep}{0pt}\setlength{\parskip}{0pt}}\usepackage{longtable,booktabs,array}
\usepackage{calc} % for calculating minipage widths
% Correct order of tables after \paragraph or \subparagraph
\usepackage{etoolbox}
\makeatletter
\patchcmd\longtable{\par}{\if@noskipsec\mbox{}\fi\par}{}{}
\makeatother
% Allow footnotes in longtable head/foot
\IfFileExists{footnotehyper.sty}{\usepackage{footnotehyper}}{\usepackage{footnote}}
\makesavenoteenv{longtable}
\usepackage{graphicx}
\makeatletter
\def\maxwidth{\ifdim\Gin@nat@width>\linewidth\linewidth\else\Gin@nat@width\fi}
\def\maxheight{\ifdim\Gin@nat@height>\textheight\textheight\else\Gin@nat@height\fi}
\makeatother
% Scale images if necessary, so that they will not overflow the page
% margins by default, and it is still possible to overwrite the defaults
% using explicit options in \includegraphics[width, height, ...]{}
\setkeys{Gin}{width=\maxwidth,height=\maxheight,keepaspectratio}
% Set default figure placement to htbp
\makeatletter
\def\fps@figure{htbp}
\makeatother
% definitions for citeproc citations
\NewDocumentCommand\citeproctext{}{}
\NewDocumentCommand\citeproc{mm}{%
  \begingroup\def\citeproctext{#2}\cite{#1}\endgroup}
\makeatletter
 % allow citations to break across lines
 \let\@cite@ofmt\@firstofone
 % avoid brackets around text for \cite:
 \def\@biblabel#1{}
 \def\@cite#1#2{{#1\if@tempswa , #2\fi}}
\makeatother
\newlength{\cslhangindent}
\setlength{\cslhangindent}{1.5em}
\newlength{\csllabelwidth}
\setlength{\csllabelwidth}{3em}
\newenvironment{CSLReferences}[2] % #1 hanging-indent, #2 entry-spacing
 {\begin{list}{}{%
  \setlength{\itemindent}{0pt}
  \setlength{\leftmargin}{0pt}
  \setlength{\parsep}{0pt}
  % turn on hanging indent if param 1 is 1
  \ifodd #1
   \setlength{\leftmargin}{\cslhangindent}
   \setlength{\itemindent}{-1\cslhangindent}
  \fi
  % set entry spacing
  \setlength{\itemsep}{#2\baselineskip}}}
 {\end{list}}
\usepackage{calc}
\newcommand{\CSLBlock}[1]{\hfill\break\parbox[t]{\linewidth}{\strut\ignorespaces#1\strut}}
\newcommand{\CSLLeftMargin}[1]{\parbox[t]{\csllabelwidth}{\strut#1\strut}}
\newcommand{\CSLRightInline}[1]{\parbox[t]{\linewidth - \csllabelwidth}{\strut#1\strut}}
\newcommand{\CSLIndent}[1]{\hspace{\cslhangindent}#1}

\makeatletter
\@ifpackageloaded{tcolorbox}{}{\usepackage[skins,breakable]{tcolorbox}}
\@ifpackageloaded{fontawesome5}{}{\usepackage{fontawesome5}}
\definecolor{quarto-callout-color}{HTML}{909090}
\definecolor{quarto-callout-note-color}{HTML}{0758E5}
\definecolor{quarto-callout-important-color}{HTML}{CC1914}
\definecolor{quarto-callout-warning-color}{HTML}{EB9113}
\definecolor{quarto-callout-tip-color}{HTML}{00A047}
\definecolor{quarto-callout-caution-color}{HTML}{FC5300}
\definecolor{quarto-callout-color-frame}{HTML}{acacac}
\definecolor{quarto-callout-note-color-frame}{HTML}{4582ec}
\definecolor{quarto-callout-important-color-frame}{HTML}{d9534f}
\definecolor{quarto-callout-warning-color-frame}{HTML}{f0ad4e}
\definecolor{quarto-callout-tip-color-frame}{HTML}{02b875}
\definecolor{quarto-callout-caution-color-frame}{HTML}{fd7e14}
\makeatother
\makeatletter
\@ifpackageloaded{bookmark}{}{\usepackage{bookmark}}
\makeatother
\makeatletter
\@ifpackageloaded{caption}{}{\usepackage{caption}}
\AtBeginDocument{%
\ifdefined\contentsname
  \renewcommand*\contentsname{Table of contents}
\else
  \newcommand\contentsname{Table of contents}
\fi
\ifdefined\listfigurename
  \renewcommand*\listfigurename{List of Figures}
\else
  \newcommand\listfigurename{List of Figures}
\fi
\ifdefined\listtablename
  \renewcommand*\listtablename{List of Tables}
\else
  \newcommand\listtablename{List of Tables}
\fi
\ifdefined\figurename
  \renewcommand*\figurename{Figure}
\else
  \newcommand\figurename{Figure}
\fi
\ifdefined\tablename
  \renewcommand*\tablename{Table}
\else
  \newcommand\tablename{Table}
\fi
}
\@ifpackageloaded{float}{}{\usepackage{float}}
\floatstyle{ruled}
\@ifundefined{c@chapter}{\newfloat{codelisting}{h}{lop}}{\newfloat{codelisting}{h}{lop}[chapter]}
\floatname{codelisting}{Listing}
\newcommand*\listoflistings{\listof{codelisting}{List of Listings}}
\makeatother
\makeatletter
\makeatother
\makeatletter
\@ifpackageloaded{caption}{}{\usepackage{caption}}
\@ifpackageloaded{subcaption}{}{\usepackage{subcaption}}
\makeatother
\ifLuaTeX
  \usepackage{selnolig}  % disable illegal ligatures
\fi
\usepackage{bookmark}

\IfFileExists{xurl.sty}{\usepackage{xurl}}{} % add URL line breaks if available
\urlstyle{same} % disable monospaced font for URLs
\hypersetup{
  pdftitle={The QPL Book},
  pdfauthor={David Coldeira},
  colorlinks=true,
  linkcolor={Maroon},
  filecolor={Maroon},
  citecolor={Blue},
  urlcolor={Blue},
  pdfcreator={LaTeX via pandoc}}

\title{The QPL Book}
\usepackage{etoolbox}
\makeatletter
\providecommand{\subtitle}[1]{% add subtitle to \maketitle
  \apptocmd{\@title}{\par {\large #1 \par}}{}{}
}
\makeatother
\subtitle{Learn Quantum Computing and MBQC with the Quantum Process
Language}
\author{David Coldeira}
\date{2026-01-04}

\begin{document}
\frontmatter
\maketitle

\renewcommand*\contentsname{Table of contents}
{
\hypersetup{linkcolor=}
\setcounter{tocdepth}{2}
\tableofcontents
}
\mainmatter
\bookmarksetup{startatroot}

\chapter*{Welcome to The QPL Book}\label{welcome-to-the-qpl-book}
\addcontentsline{toc}{chapter}{Welcome to The QPL Book}

\markboth{Welcome to The QPL Book}{Welcome to The QPL Book}

Learn quantum computing and Measurement-Based Quantum Computing (MBQC)
using the \textbf{Quantum Process Language} (QPL) - a relations-first
approach to quantum programming.

\section*{What You'll Learn}\label{what-youll-learn}
\addcontentsline{toc}{section}{What You'll Learn}

\markright{What You'll Learn}

This book teaches quantum computing from first principles, building up
to advanced topics in MBQC and photonic quantum computing:

\begin{itemize}
\tightlist
\item
  \textbf{Part I: Quantum Foundations} - Understand qubits,
  entanglement, and n-qubit systems
\item
  \textbf{Part II: MBQC Theory} - Learn cluster states, graph states,
  and measurement patterns
\item
  \textbf{Part III: QPL Programming} - Write quantum programs with
  hands-on tutorials
\item
  \textbf{Part IV: Advanced Topics} - Explore tensor networks,
  categorical quantum mechanics, and fault-tolerant QC
\end{itemize}

\section*{Why QPL?}\label{why-qpl}
\addcontentsline{toc}{section}{Why QPL?}

\markright{Why QPL?}

Unlike gate-based quantum languages (Qiskit, Cirq, Q\#), QPL treats
\textbf{quantum entanglement as a first-class primitive}, making it
natural to express:

\begin{itemize}
\tightlist
\item
  Cluster states for MBQC
\item
  Measurement patterns for photonic quantum computers
\item
  Relations-first quantum algorithms
\item
  Graph state compilation
\end{itemize}

\section*{Prerequisites}\label{prerequisites}
\addcontentsline{toc}{section}{Prerequisites}

\markright{Prerequisites}

\begin{itemize}
\tightlist
\item
  \textbf{Mathematics}: Linear algebra, complex numbers, basic
  probability
\item
  \textbf{Programming}: Python basics (QPL is implemented in Python)
\item
  \textbf{Physics}: Helpful but not required - we explain quantum
  mechanics from scratch
\end{itemize}

\section*{How to Use This Book}\label{how-to-use-this-book}
\addcontentsline{toc}{section}{How to Use This Book}

\markright{How to Use This Book}

\subsection*{For Learners}\label{for-learners}
\addcontentsline{toc}{subsection}{For Learners}

Start with Part I to build quantum foundations, then move through
hands-on tutorials in Part III.

\subsection*{For Researchers}\label{for-researchers}
\addcontentsline{toc}{subsection}{For Researchers}

Jump to Part II for MBQC theory or Part IV for advanced topics like
tensor networks.

\subsection*{For Programmers}\label{for-programmers}
\addcontentsline{toc}{subsection}{For Programmers}

Install QPL and dive into Part III tutorials - learn by doing!

\section*{Installation}\label{installation}
\addcontentsline{toc}{section}{Installation}

\markright{Installation}

\begin{Shaded}
\begin{Highlighting}[]
\ExtensionTok{pip}\NormalTok{ install quantum{-}process{-}language}
\end{Highlighting}
\end{Shaded}

See \href{appendix/installation.qmd}{Appendix: Installation} for
detailed setup instructions.

\section*{Code Examples}\label{code-examples}
\addcontentsline{toc}{section}{Code Examples}

\markright{Code Examples}

All code in this book is: - \textbf{Executable} - Run examples directly
in your Python environment - \textbf{Open source} - Available at
\href{https://github.com/dcoldeira/quantum-process-language}{github.com/dcoldeira/quantum-process-language}
- \textbf{Tested} - Every example is verified to work with the latest
QPL release

\section*{About the Author}\label{about-the-author}
\addcontentsline{toc}{section}{About the Author}

\markright{About the Author}

\textbf{David Coldeira} is a scientific software engineer with a BSc in
Physics, specializing in quantum programming languages and MBQC. He
developed QPL to explore relations-first approaches to quantum
computing.

\begin{itemize}
\tightlist
\item
  Email: dcoldeira@gmail.com
\item
  GitHub: \href{https://github.com/dcoldeira}{(\textbf{dcoldeira?})}
\item
  Blog: \href{https://dcoldeira.github.io}{dcoldeira.github.io}
\end{itemize}

\section*{Contributing}\label{contributing}
\addcontentsline{toc}{section}{Contributing}

\markright{Contributing}

Found a typo? Have a suggestion? Contributions are welcome!

\begin{itemize}
\tightlist
\item
  \textbf{Issues}:
  \href{https://github.com/dcoldeira/qpl-book/issues}{github.com/dcoldeira/qpl-book/issues}
\item
  \textbf{Pull requests}:
  \href{https://github.com/dcoldeira/qpl-book}{github.com/dcoldeira/qpl-book}
\end{itemize}

\section*{License}\label{license}
\addcontentsline{toc}{section}{License}

\markright{License}

\begin{itemize}
\tightlist
\item
  \textbf{Text}: Creative Commons BY-SA 4.0
\item
  \textbf{Code}: MIT License (same as QPL)
\end{itemize}

\begin{center}\rule{0.5\linewidth}{0.5pt}\end{center}

\textbf{Let's begin!} Start with
\href{chapters/01-quantum-basics.qmd}{Chapter 1: Quantum Mechanics
Basics} or jump straight to hands-on tutorials in
\href{chapters/07-first-bell-state.qmd}{Chapter 7: Your First Bell
State}.

\part{Part I: Quantum Foundations}

\chapter{Chapter 01}\label{chapter-01}

\emph{This chapter is coming soon\ldots{}}

\chapter{Chapter 02}\label{chapter-02}

\emph{This chapter is coming soon\ldots{}}

\chapter{Chapter 03}\label{chapter-03}

\emph{This chapter is coming soon\ldots{}}

\part{Part II: Measurement-Based Quantum Computing}

\chapter{Chapter 04}\label{chapter-04}

\emph{This chapter is coming soon\ldots{}}

\chapter{Chapter 05}\label{chapter-05}

\emph{This chapter is coming soon\ldots{}}

\chapter{Chapter 06}\label{chapter-06}

\emph{This chapter is coming soon\ldots{}}

\part{Part III: QPL Programming}

\chapter{Your First Bell State}\label{your-first-bell-state}

Welcome to your first hands-on QPL tutorial! In this chapter, you'll
create your first quantum program: a Bell state - one of the most
fundamental entangled quantum states.

\section{What is a Bell State?}\label{what-is-a-bell-state}

A \textbf{Bell state} is a maximally entangled 2-qubit quantum state.
The most common Bell state is:

\[
|\Phi^+\rangle = \frac{|00\rangle + |11\rangle}{\sqrt{2}}
\]

This state has fascinating properties: - \textbf{Superposition}: Equal
probability of measuring \(|00\rangle\) or \(|11\rangle\) -
\textbf{Perfect correlation}: If you measure qubit A as 0, qubit B will
\textbf{always} be 0 - \textbf{Maximal entanglement}: Entropy
\(S = 1.0\) (maximum for 2 qubits)

\section{Setting Up}\label{setting-up}

First, make sure QPL is installed:

\begin{Shaded}
\begin{Highlighting}[]
\ExtensionTok{pip}\NormalTok{ install quantum{-}process{-}language}
\end{Highlighting}
\end{Shaded}

Create a new Python file \texttt{my\_first\_bell\_state.py}:

\begin{Shaded}
\begin{Highlighting}[]
\ImportTok{from}\NormalTok{ qpl }\ImportTok{import}\NormalTok{ QPLProgram, create\_question, QuestionType}

\CommentTok{\# Create a quantum program}
\NormalTok{program }\OperatorTok{=}\NormalTok{ QPLProgram(}\StringTok{"My First Bell State"}\NormalTok{)}
\end{Highlighting}
\end{Shaded}

\section{Creating Quantum Systems}\label{creating-quantum-systems}

In QPL, you don't create individual ``qubits'' - you create
\textbf{quantum systems} that can then be entangled into
\textbf{relations}:

\begin{Shaded}
\begin{Highlighting}[]
\CommentTok{\# Create two quantum systems}
\NormalTok{qubit\_a }\OperatorTok{=}\NormalTok{ program.create\_system()}
\NormalTok{qubit\_b }\OperatorTok{=}\NormalTok{ program.create\_system()}

\BuiltInTok{print}\NormalTok{(}\SpecialStringTok{f"Created systems: }\SpecialCharTok{\{}\NormalTok{qubit\_a}\SpecialCharTok{.}\NormalTok{system\_id}\SpecialCharTok{\}}\SpecialStringTok{ and }\SpecialCharTok{\{}\NormalTok{qubit\_b}\SpecialCharTok{.}\NormalTok{system\_id}\SpecialCharTok{\}}\SpecialStringTok{"}\NormalTok{)}
\end{Highlighting}
\end{Shaded}

Each system gets a unique ID. At this point, they're independent - not
yet entangled.

\section{Entangling Systems}\label{entangling-systems}

Here's where QPL's philosophy shines. Instead of applying gates like
\texttt{CNOT}, you directly express the \textbf{relationship} you want:

\begin{Shaded}
\begin{Highlighting}[]
\CommentTok{\# Create a Bell pair (maximally entangled state)}
\NormalTok{bell\_pair }\OperatorTok{=}\NormalTok{ program.entangle(qubit\_a, qubit\_b)}

\BuiltInTok{print}\NormalTok{(}\SpecialStringTok{f"Bell state created!"}\NormalTok{)}
\BuiltInTok{print}\NormalTok{(}\SpecialStringTok{f"State vector shape: }\SpecialCharTok{\{}\NormalTok{bell\_pair}\SpecialCharTok{.}\NormalTok{state}\SpecialCharTok{.}\NormalTok{shape}\SpecialCharTok{\}}\SpecialStringTok{"}\NormalTok{)}
\BuiltInTok{print}\NormalTok{(}\SpecialStringTok{f"Entanglement entropy: }\SpecialCharTok{\{}\NormalTok{bell\_pair}\SpecialCharTok{.}\NormalTok{entanglement\_entropy}\SpecialCharTok{:.3f\}}\SpecialStringTok{"}\NormalTok{)}
\end{Highlighting}
\end{Shaded}

\textbf{Output:}

\begin{verbatim}
Bell state created!
State vector shape: (4,)
Entanglement entropy: 1.000
\end{verbatim}

The entropy of 1.0 confirms this is \textbf{maximally entangled}.

\section{Understanding the State}\label{understanding-the-state}

Let's look at the actual quantum state:

\begin{Shaded}
\begin{Highlighting}[]
\ImportTok{import}\NormalTok{ numpy }\ImportTok{as}\NormalTok{ np}

\BuiltInTok{print}\NormalTok{(}\StringTok{"Bell state vector:"}\NormalTok{)}
\BuiltInTok{print}\NormalTok{(bell\_pair.state)}
\BuiltInTok{print}\NormalTok{()}
\BuiltInTok{print}\NormalTok{(}\StringTok{"In basis notation:"}\NormalTok{)}
\BuiltInTok{print}\NormalTok{(}\SpecialStringTok{f"|00⟩: }\SpecialCharTok{\{}\NormalTok{bell\_pair}\SpecialCharTok{.}\NormalTok{state[}\DecValTok{0}\NormalTok{]}\SpecialCharTok{:.3f\}}\SpecialStringTok{"}\NormalTok{)}
\BuiltInTok{print}\NormalTok{(}\SpecialStringTok{f"|01⟩: }\SpecialCharTok{\{}\NormalTok{bell\_pair}\SpecialCharTok{.}\NormalTok{state[}\DecValTok{1}\NormalTok{]}\SpecialCharTok{:.3f\}}\SpecialStringTok{"}\NormalTok{)}
\BuiltInTok{print}\NormalTok{(}\SpecialStringTok{f"|10⟩: }\SpecialCharTok{\{}\NormalTok{bell\_pair}\SpecialCharTok{.}\NormalTok{state[}\DecValTok{2}\NormalTok{]}\SpecialCharTok{:.3f\}}\SpecialStringTok{"}\NormalTok{)}
\BuiltInTok{print}\NormalTok{(}\SpecialStringTok{f"|11⟩: }\SpecialCharTok{\{}\NormalTok{bell\_pair}\SpecialCharTok{.}\NormalTok{state[}\DecValTok{3}\NormalTok{]}\SpecialCharTok{:.3f\}}\SpecialStringTok{"}\NormalTok{)}
\end{Highlighting}
\end{Shaded}

\textbf{Output:}

\begin{verbatim}
Bell state vector:
[0.707+0j  0.+0j  0.+0j  0.707+0j]

In basis notation:
|00⟩: 0.707
|01⟩: 0.000
|10⟩: 0.000
|11⟩: 0.707
\end{verbatim}

This is exactly \(\frac{|00\rangle + |11\rangle}{\sqrt{2}}\)! (Note:
\(0.707 \approx \frac{1}{\sqrt{2}}\))

\section{Measuring the Bell State}\label{measuring-the-bell-state}

Now let's ask a \textbf{question} of the quantum system - in QPL,
measurement is asking a question:

\begin{Shaded}
\begin{Highlighting}[]
\CommentTok{\# Create a measurement question (Z{-}basis)}
\NormalTok{question\_z }\OperatorTok{=}\NormalTok{ create\_question(QuestionType.SPIN\_Z)}

\CommentTok{\# Add a perspective (observer)}
\NormalTok{alice }\OperatorTok{=}\NormalTok{ program.add\_perspective(}\StringTok{"alice"}\NormalTok{, \{}\StringTok{"role"}\NormalTok{: }\StringTok{"experimenter"}\NormalTok{\})}

\CommentTok{\# Ask the question {-} this performs measurement}
\NormalTok{result }\OperatorTok{=}\NormalTok{ program.ask(bell\_pair, question\_z, perspective}\OperatorTok{=}\StringTok{"alice"}\NormalTok{)}

\BuiltInTok{print}\NormalTok{(}\SpecialStringTok{f"Measurement result: }\SpecialCharTok{\{}\NormalTok{result}\SpecialCharTok{\}}\SpecialStringTok{"}\NormalTok{)}
\BuiltInTok{print}\NormalTok{(}\SpecialStringTok{f"State after measurement: }\SpecialCharTok{\{}\NormalTok{bell\_pair}\SpecialCharTok{.}\NormalTok{state}\SpecialCharTok{\}}\SpecialStringTok{"}\NormalTok{)}
\end{Highlighting}
\end{Shaded}

\textbf{Possible output:}

\begin{verbatim}
Measurement result: {'system_0': 0, 'system_1': 0}
State after measurement: [1.+0j 0.+0j 0.+0j 0.+0j]
\end{verbatim}

Or:

\begin{verbatim}
Measurement result: {'system_0': 1, 'system_1': 1}
State after measurement: [0.+0j 0.+0j 0.+0j 1.+0j]
\end{verbatim}

Notice: \textbf{Both qubits always have the same outcome!} This is the
Bell correlation.

\section{Verifying Bell Correlations}\label{verifying-bell-correlations}

Let's measure many times and verify the correlation:

\begin{Shaded}
\begin{Highlighting}[]
\KeywordTok{def}\NormalTok{ test\_bell\_correlations(num\_trials}\OperatorTok{=}\DecValTok{100}\NormalTok{):}
    \CommentTok{"""Run multiple trials to verify Bell state correlations."""}

\NormalTok{    same\_count }\OperatorTok{=} \DecValTok{0}  \CommentTok{\# How many times both qubits match}

    \ControlFlowTok{for}\NormalTok{ trial }\KeywordTok{in} \BuiltInTok{range}\NormalTok{(num\_trials):}
        \CommentTok{\# Create fresh Bell state}
\NormalTok{        program }\OperatorTok{=}\NormalTok{ QPLProgram(}\SpecialStringTok{f"Bell Trial }\SpecialCharTok{\{}\NormalTok{trial}\SpecialCharTok{\}}\SpecialStringTok{"}\NormalTok{)}
\NormalTok{        q0 }\OperatorTok{=}\NormalTok{ program.create\_system()}
\NormalTok{        q1 }\OperatorTok{=}\NormalTok{ program.create\_system()}
\NormalTok{        bell }\OperatorTok{=}\NormalTok{ program.entangle(q0, q1)}

        \CommentTok{\# Measure in Z basis}
\NormalTok{        question }\OperatorTok{=}\NormalTok{ create\_question(QuestionType.SPIN\_Z)}
\NormalTok{        alice }\OperatorTok{=}\NormalTok{ program.add\_perspective(}\StringTok{"alice"}\NormalTok{)}
\NormalTok{        result }\OperatorTok{=}\NormalTok{ program.ask(bell, question, perspective}\OperatorTok{=}\StringTok{"alice"}\NormalTok{)}

        \CommentTok{\# Check if outcomes match}
\NormalTok{        outcomes }\OperatorTok{=} \BuiltInTok{list}\NormalTok{(result.values())}
        \ControlFlowTok{if}\NormalTok{ outcomes[}\DecValTok{0}\NormalTok{] }\OperatorTok{==}\NormalTok{ outcomes[}\DecValTok{1}\NormalTok{]:}
\NormalTok{            same\_count }\OperatorTok{+=} \DecValTok{1}

\NormalTok{    correlation }\OperatorTok{=}\NormalTok{ same\_count }\OperatorTok{/}\NormalTok{ num\_trials}
    \BuiltInTok{print}\NormalTok{(}\SpecialStringTok{f"Trials: }\SpecialCharTok{\{}\NormalTok{num\_trials}\SpecialCharTok{\}}\SpecialStringTok{"}\NormalTok{)}
    \BuiltInTok{print}\NormalTok{(}\SpecialStringTok{f"Both qubits matched: }\SpecialCharTok{\{}\NormalTok{same\_count}\SpecialCharTok{\}}\SpecialStringTok{/}\SpecialCharTok{\{}\NormalTok{num\_trials}\SpecialCharTok{\}}\SpecialStringTok{"}\NormalTok{)}
    \BuiltInTok{print}\NormalTok{(}\SpecialStringTok{f"Correlation: }\SpecialCharTok{\{}\NormalTok{correlation}\SpecialCharTok{:.1\%\}}\SpecialStringTok{"}\NormalTok{)}
    \BuiltInTok{print}\NormalTok{()}
    \BuiltInTok{print}\NormalTok{(}\StringTok{"Expected: 100}\SpecialCharTok{\% c}\StringTok{orrelation for Bell state"}\NormalTok{)}

\CommentTok{\# Run the test}
\NormalTok{test\_bell\_correlations(}\DecValTok{100}\NormalTok{)}
\end{Highlighting}
\end{Shaded}

\textbf{Output:}

\begin{verbatim}
Trials: 100
Both qubits matched: 100/100
Correlation: 100.0%

Expected: 100% correlation for Bell state
\end{verbatim}

\textbf{Perfect correlation!} This is quantum entanglement in action.

\section{Cross-Basis Measurement}\label{cross-basis-measurement}

What happens if we measure in different bases?

\begin{Shaded}
\begin{Highlighting}[]
\CommentTok{\# Create Bell state}
\NormalTok{program }\OperatorTok{=}\NormalTok{ QPLProgram(}\StringTok{"Cross{-}Basis Measurement"}\NormalTok{)}
\NormalTok{q0 }\OperatorTok{=}\NormalTok{ program.create\_system()}
\NormalTok{q1 }\OperatorTok{=}\NormalTok{ program.create\_system()}
\NormalTok{bell }\OperatorTok{=}\NormalTok{ program.entangle(q0, q1)}

\CommentTok{\# Measure in X basis (Hadamard basis)}
\NormalTok{question\_x }\OperatorTok{=}\NormalTok{ create\_question(QuestionType.SPIN\_X)}
\NormalTok{alice }\OperatorTok{=}\NormalTok{ program.add\_perspective(}\StringTok{"alice"}\NormalTok{)}

\NormalTok{result\_x }\OperatorTok{=}\NormalTok{ program.ask(bell, question\_x, perspective}\OperatorTok{=}\StringTok{"alice"}\NormalTok{)}
\BuiltInTok{print}\NormalTok{(}\SpecialStringTok{f"X{-}basis measurement: }\SpecialCharTok{\{}\NormalTok{result\_x}\SpecialCharTok{\}}\SpecialStringTok{"}\NormalTok{)}
\end{Highlighting}
\end{Shaded}

In X-basis, Bell states \textbf{still} show perfect correlation! Try it
yourself.

\section{Complete Example}\label{complete-example}

Here's the complete program you can run:

\begin{Shaded}
\begin{Highlighting}[]
\ImportTok{from}\NormalTok{ qpl }\ImportTok{import}\NormalTok{ QPLProgram, create\_question, QuestionType}

\KeywordTok{def}\NormalTok{ main():}
    \CommentTok{\# Create program}
\NormalTok{    program }\OperatorTok{=}\NormalTok{ QPLProgram(}\StringTok{"Complete Bell State Example"}\NormalTok{)}

    \CommentTok{\# Create quantum systems}
\NormalTok{    qubit\_a }\OperatorTok{=}\NormalTok{ program.create\_system()}
\NormalTok{    qubit\_b }\OperatorTok{=}\NormalTok{ program.create\_system()}

    \CommentTok{\# Entangle into Bell state}
\NormalTok{    bell\_pair }\OperatorTok{=}\NormalTok{ program.entangle(qubit\_a, qubit\_b)}

    \CommentTok{\# Show state information}
    \BuiltInTok{print}\NormalTok{(}\StringTok{"="} \OperatorTok{*} \DecValTok{50}\NormalTok{)}
    \BuiltInTok{print}\NormalTok{(}\StringTok{"BELL STATE CREATED"}\NormalTok{)}
    \BuiltInTok{print}\NormalTok{(}\StringTok{"="} \OperatorTok{*} \DecValTok{50}\NormalTok{)}
    \BuiltInTok{print}\NormalTok{(}\SpecialStringTok{f"State vector: }\SpecialCharTok{\{}\NormalTok{bell\_pair}\SpecialCharTok{.}\NormalTok{state}\SpecialCharTok{\}}\SpecialStringTok{"}\NormalTok{)}
    \BuiltInTok{print}\NormalTok{(}\SpecialStringTok{f"Entanglement entropy: }\SpecialCharTok{\{}\NormalTok{bell\_pair}\SpecialCharTok{.}\NormalTok{entanglement\_entropy}\SpecialCharTok{:.3f\}}\SpecialStringTok{"}\NormalTok{)}
    \BuiltInTok{print}\NormalTok{(}\SpecialStringTok{f"Systems: }\SpecialCharTok{\{}\BuiltInTok{list}\NormalTok{(bell\_pair.systems)}\SpecialCharTok{\}}\SpecialStringTok{"}\NormalTok{)}
    \BuiltInTok{print}\NormalTok{()}

    \CommentTok{\# Measure}
\NormalTok{    question\_z }\OperatorTok{=}\NormalTok{ create\_question(QuestionType.SPIN\_Z)}
\NormalTok{    alice }\OperatorTok{=}\NormalTok{ program.add\_perspective(}\StringTok{"alice"}\NormalTok{, \{}\StringTok{"role"}\NormalTok{: }\StringTok{"experimenter"}\NormalTok{\})}

\NormalTok{    result }\OperatorTok{=}\NormalTok{ program.ask(bell\_pair, question\_z, perspective}\OperatorTok{=}\StringTok{"alice"}\NormalTok{)}

    \BuiltInTok{print}\NormalTok{(}\StringTok{"="} \OperatorTok{*} \DecValTok{50}\NormalTok{)}
    \BuiltInTok{print}\NormalTok{(}\StringTok{"MEASUREMENT RESULT"}\NormalTok{)}
    \BuiltInTok{print}\NormalTok{(}\StringTok{"="} \OperatorTok{*} \DecValTok{50}\NormalTok{)}
    \BuiltInTok{print}\NormalTok{(}\SpecialStringTok{f"Measured outcomes: }\SpecialCharTok{\{}\NormalTok{result}\SpecialCharTok{\}}\SpecialStringTok{"}\NormalTok{)}
    \BuiltInTok{print}\NormalTok{(}\SpecialStringTok{f"State after collapse: }\SpecialCharTok{\{}\NormalTok{bell\_pair}\SpecialCharTok{.}\NormalTok{state}\SpecialCharTok{\}}\SpecialStringTok{"}\NormalTok{)}
    \BuiltInTok{print}\NormalTok{()}

    \CommentTok{\# Verify correlation}
\NormalTok{    outcomes }\OperatorTok{=} \BuiltInTok{list}\NormalTok{(result.values())}
    \ControlFlowTok{if}\NormalTok{ outcomes[}\DecValTok{0}\NormalTok{] }\OperatorTok{==}\NormalTok{ outcomes[}\DecValTok{1}\NormalTok{]:}
        \BuiltInTok{print}\NormalTok{(}\StringTok{"✓ Both qubits have SAME outcome (correlated!)"}\NormalTok{)}
    \ControlFlowTok{else}\NormalTok{:}
        \BuiltInTok{print}\NormalTok{(}\StringTok{"✗ Qubits have DIFFERENT outcomes (not possible for Bell state)"}\NormalTok{)}

\ControlFlowTok{if} \VariableTok{\_\_name\_\_} \OperatorTok{==} \StringTok{"\_\_main\_\_"}\NormalTok{:}
\NormalTok{    main()}
\end{Highlighting}
\end{Shaded}

Save this as \texttt{my\_first\_bell\_state.py} and run:

\begin{Shaded}
\begin{Highlighting}[]
\ExtensionTok{python}\NormalTok{ my\_first\_bell\_state.py}
\end{Highlighting}
\end{Shaded}

\section{What You Learned}\label{what-you-learned}

Congratulations! You've just:

✅ Created your first quantum program in QPL ✅ Generated a maximally
entangled Bell state ✅ Measured quantum correlations ✅ Verified
perfect correlation (100\%) ✅ Understood entanglement entropy

\section{Key Concepts}\label{key-concepts}

\begin{enumerate}
\def\labelenumi{\arabic{enumi}.}
\tightlist
\item
  \textbf{Relations over objects}: You created a
  \texttt{QuantumRelation}, not individual qubits
\item
  \textbf{Questions over measurements}: You used \texttt{ask()}, not
  \texttt{measure()}
\item
  \textbf{Perspectives}: Observers (\texttt{alice}) are explicit in QPL
\item
  \textbf{Entanglement first}: \texttt{entangle()} is a primitive
  operation, not derived from gates
\end{enumerate}

\section{Next Steps}\label{next-steps}

Ready for more? Try:

\begin{itemize}
\tightlist
\item
  \textbf{Chapter 8: Understanding GHZ States} - Extend to 3+ qubits
\item
  \textbf{Chapter 9: Measurement in QPL} - Explore different measurement
  bases
\item
  \textbf{Chapter 10: Quantum Teleportation} - Use Bell states to
  teleport quantum information
\end{itemize}

\section{Exercises}\label{exercises}

\begin{enumerate}
\def\labelenumi{\arabic{enumi}.}
\item
  \textbf{Modify the code} to measure in X-basis instead of Z-basis. Do
  correlations still hold?
\item
  \textbf{Create all 4 Bell states}:

  \begin{itemize}
  \tightlist
  \item
    \(|\Phi^+\rangle = \frac{|00\rangle + |11\rangle}{\sqrt{2}}\) (what
    you just made)
  \item
    \(|\Phi^-\rangle = \frac{|00\rangle - |11\rangle}{\sqrt{2}}\)
  \item
    \(|\Psi^+\rangle = \frac{|01\rangle + |10\rangle}{\sqrt{2}}\)
  \item
    \(|\Psi^-\rangle = \frac{|01\rangle - |10\rangle}{\sqrt{2}}\)
  \end{itemize}

  Hint: Check the QPL documentation for \texttt{entangle()} options.
\item
  \textbf{Statistical experiment}: Run 1000 trials and plot the
  distribution of measurement outcomes. What do you observe?
\item
  \textbf{Partial measurement}: What happens if you measure just one
  qubit? Implement this and observe the state collapse.
\end{enumerate}

\begin{tcolorbox}[enhanced jigsaw, rightrule=.15mm, toptitle=1mm, colbacktitle=quarto-callout-tip-color!10!white, left=2mm, opacitybacktitle=0.6, colframe=quarto-callout-tip-color-frame, breakable, leftrule=.75mm, coltitle=black, toprule=.15mm, bottomtitle=1mm, bottomrule=.15mm, titlerule=0mm, opacityback=0, title=\textcolor{quarto-callout-tip-color}{\faLightbulb}\hspace{0.5em}{Going Deeper}, arc=.35mm, colback=white]

Want to understand the physics behind Bell states? See \textbf{Chapter
2: Entanglement} for the theoretical foundation.

Want to see how QPL compares to Qiskit? Check out the
\href{https://dcoldeira.github.io/posts/qpl-vs-qiskit/}{QPL vs Qiskit
comparison} on the blog.

\end{tcolorbox}

\begin{center}\rule{0.5\linewidth}{0.5pt}\end{center}

\textbf{Next:} \href{08-ghz-states.qmd}{Chapter 8: Understanding GHZ
States} - Scale to 3+ qubits

\chapter{Chapter 08}\label{chapter-08}

\emph{This chapter is coming soon\ldots{}}

\chapter{Chapter 09}\label{chapter-09}

\emph{This chapter is coming soon\ldots{}}

\chapter{Chapter 10}\label{chapter-10}

\emph{This chapter is coming soon\ldots{}}

\chapter{Chapter 11}\label{chapter-11}

\emph{This chapter is coming soon\ldots{}}

\part{Part IV: Advanced Topics}

\chapter{Chapter 12}\label{chapter-12}

\emph{This chapter is coming soon\ldots{}}

\chapter{Chapter 13}\label{chapter-13}

\emph{This chapter is coming soon\ldots{}}

\chapter{Chapter 14}\label{chapter-14}

\emph{This chapter is coming soon\ldots{}}

\chapter{Chapter 15}\label{chapter-15}

\emph{This chapter is coming soon\ldots{}}

\part{Appendices}

\chapter{Installation}\label{installation-1}

\emph{Coming soon\ldots{}}

\chapter{API Reference}\label{api-reference}

\emph{Coming soon\ldots{}}

\chapter*{References}\label{references}
\addcontentsline{toc}{chapter}{References}

\markboth{References}{References}

\phantomsection\label{refs}
\begin{CSLReferences}{0}{1}
\end{CSLReferences}


\backmatter

\end{document}
